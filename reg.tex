\begin{frame}{Biblio Radiative transfer}
$\S$ Mihalas
– "Stellar Atmospheres" (1970) (?)
– "Stellar Atmospheres" (1978) (*)
– with Hubeny ́: "Theory of Stellar Atmospheres" (2014)
$\S$ simpler
– Novotny: "Introduction to stellar atmospheres and interiors" (1973)
– Gray: "The observation and analysis of stellar photospheres" (2005) – Bo ̈hm-Vitense: "Introduction to stellar astrophysics I, II" (1989)
$\S$ harder
– Cannon: "The transfer of spectral line radiation" (1985)
– Mihalas & Mihalas: "Foundations of Radiation Hydrodynamics" (1984) (*) – Castor: "Radiation Hydrodynamics" (2004)
$\S$ my stuff
– IART: bachelors-level radiative transfer (1991, 2015, 20??) (*)
– RTSA: masters-level Mihalas popularization (2003, 20??) (*)
– ISSF: bachelors-level introduction to NLTE (1993) (*)
– Monterey: PhD-level refresher NLTE chromospheric lines for IRIS (2012) (*) – Cartagena: tutorial non-E hydrogen diagnostics for ALMA (2016) (*)
\end{frame}

\begin{frame}[allowframebreaks]{Reg Lez (rec)}

\begin{itemize}
  

\item 8 ottobre. armonic oscillator one characteristic frequency-planck function used in E. coefficient  (TE);broaden due to spontaneous decay: $A\propto\Gamma$. Lorentzian profile. Complete redistribution: efficience of absorption do not affect re-emission.
Seeing through optical thin medium sampling different temperature. Rotational profile. Voigt profile $H(\frac{\Delta\omea}{\Delta\omega_D},\frac{\Gamma}{\Delta\omega_D})=H(v,a)$. Equivalent width (line). Regime in cui le ali sono trascurabili: small a. Curve of growth.

\end{itemize}

\end{frame}