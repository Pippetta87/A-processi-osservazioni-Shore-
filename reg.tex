\begin{frame}{Biblio Radiative transfer}
$\S$ Mihalas
– "Stellar Atmospheres" (1970) (?)
– "Stellar Atmospheres" (1978) (*)
– with Hubeny : "Theory of Stellar Atmospheres" (2014)
$\S$ simpler
– Novotny: "Introduction to stellar atmospheres and interiors" (1973)
– Gray: "The observation and analysis of stellar photospheres" (2005) – Bohm-Vitense: "Introduction to stellar astrophysics I, II" (1989)
$\S$ harder
– Cannon: "The transfer of spectral line radiation" (1985)
– Mihalas $\&$ Mihalas: "Foundations of Radiation Hydrodynamics" (1984) (*) – Castor: "Radiation Hydrodynamics" (2004)
$\S$ my stuff
– IART: bachelors-level radiative transfer (1991, 2015, 20??) (*)
– RTSA: masters-level Mihalas popularization (2003, 20??) (*)
– ISSF: bachelors-level introduction to NLTE (1993) (*)
– Monterey: PhD-level refresher NLTE chromospheric lines for IRIS (2012) (*) – Cartagena: tutorial non-E hydrogen diagnostics for ALMA (2016) (*)
\end{frame}

\begin{frame}[allowframebreaks]{Reg Lez (rec)}

\begin{itemize}
\item 4 Ottobre. P cygni spectrum: frobbidden/permitted lines, broadening, shock, autoionization, velocity gradient.
LTE, coupling rules, line ratio (ration intensity members of multiplets),

 map wavelength to pixels, Stati atomici,
\item 8 ottobre. (Processi) armonic oscillator one characteristic frequency-planck function used in E. coefficient  (TE);broaden due to spontaneous decay: $A\propto\Gamma$. Lorentzian profile. Complete redistribution: efficience of absorption do not affect re-emission.
Seeing through optical thin medium sampling different temperature. Rotational profile. Voigt profile $H(\frac{\Delta\omega}{\Delta\omega_D},\frac{\Gamma}{\Delta\omega_D})=H(v,a)$. Equivalent width (line). Regime in cui le ali sono trascurabili: small a. Curve of growth.
\item 9 Ott (Osservativa). Colori: Bande fotometriche Johnson, Str\:omgren. Interferenza: separazione ordine (dispersion). Fotometria: opacit\'a atmosfera vs spazio; osservazione relative da Terra (bootstrapping) (?MK system); plate vs CCD; Vega vs AB-mag. Atmosfera: real/immaginary part of extintion; horizon problem; fluctuation of index of refraction (seeing): lunghezza di coerenza, turbulence. PSF: Gaussian (uncorrelate fluctuation). RA/DEC and field rotation.

\item Lun mart: resonant de-excitation, molecules

\item 18 Ottobre. CCD; low resolution spectra; strumental noise (temerature), ...
Frank-condon factor; molecular reaction
thermal electric fluctuation
\item 29 ottobre. Frank-Condon facor. Electron configuration in molecules. H molecular almost transparent. Radiative transfer; atmosphere: albedo, radiative equilibrium, atmosphere, star wind.
continuum media vs cosmological conditions: 21cm line
\item 30 ottobre. Astrofisica: come si passa da descrizione miscroscopica a macroscopica? (galaxies-fluid + poisson eq); collisional/less: momentum transfer, kinetic (dis)equilibrium, E/L description (c.e.). Coupling between mass, momentum, energy conservation: hierarchy of momenta. Duck Lion. Intro to turbulence. Stress tensor: shear is the non-diagonal term. Shock: discontinuity surface (spatial scale go to zero). continuous conditions across shock: shock adiabat ($\textasciicircum$). Discontinuity: casello dove la densit\'a dopo \'e minore di quella prima. How we can define T after shock front passage. Precursor. Ionization front, Str\"omgren sphere.

\item mon/tue (5-6/11) ???
Interferometry, fossil, HII region shock(?),trinity: self-similar solution time since ignition, dimension what is energy?
Blast solution

\item 8/11. shock front, selfsimilarity, HII regions, molecular cloud, barotropic shock front become corrugate: we can't mantains self-similar solutions; precursor. Problem: $V(x)$, $D(x)$, $\Pi(x)$ functions of $M_0$ (Mach number) at $x=1$.
wind: mechanical luminosity (work done on environment),

\item 12/11. Evaporation, outflow: solution (Bondi-Littleton-Hoyle ('49), Parker ('66)). Eddington limit (AGN/accretion around BH). Diffusion vs wind; lines. Milne instability in stellar wind: Milne '27, Rybyhky Owacky '80s, Lucy white '80s, Lucy Solomon '80s. P Cygni profile: Mihalas and Mihalas '84, Hubeny Mihalas '16

\item 13/11. star cluster simulation: mass (RV 1/f) $->$ IMF $->$ LF(m), distribute star into grid (100x100), sorting algorithm, noise, recover knonws in a field; completeness. PSF. Decompose vs deconvolve. (2 weeks ago: noise inside detector multiplicate: interferometer   crruciality)

\item 15/11. radiative acceleration. isotropy-OpticallyThick-dense-EddingtonFactor, Thompson electron opacity; element diffusion: Hg-Mn stars.

Outflow problem: velocity field assumption (we solve radiative transfer not dynamics). 
Jets, de Lavel problem. Buoyancy. Rayleigh-Taylor instability.

\item 19/11. (pavlek seminario: galactic halos, ...) Buoyancy instability: cannot drink chinotto falling from leaning tower, Rayleigh-Taylor instability; shear $->$ vorticity; Kelvin-Heltzmholtz instability; (equilibrium is a instantaneous statement) ''act like viscosity'': turbulence. Mixing length: prandtl vs bierman -vitense; $\alpha\propo H_p$. ''If anything worth bathroom reading something on fluid it did''.

\item 20/11. Spectrograph: grating, gazed grating, the flux through the slit produce divergent beam \so{} we use lense/mirror to stir up the flux, optical fiber, \'echelle spectrograph, gr-sm; line spread function (optics+detector), refraction order; PSF (optics+detector+atmosphere): Airy disk. Time-series analysis: Nyquist frequency, periodic signals.

\item 22/11. Convection: CNO vs PP stars, Ledoux criterion, energy transport. Rayleigh-Benard convection, Raynold decomposition, fully developed convection, characteristic length: Rayleigh number.

\item mar. Jeans-problem: ??

\item 29/11. Jeans instability: perturbation around hydrostatic equilibrium, we know what start the process not what stop it, in thermal system T may rise: opacity limited collapse when we may have transition between isothermal to adiabatic; energy loss: resonant lines (atomic) opacity is irrelevant (few eV) - in molecular clouds T=meV, hard to stop cooling since when density rise collisional excitation rise: if density high enough we have collisional de-excitation and we become adiabatic. 
opacity and collisional de-excitation at T=30K. Accretion disk: viscosity, stress tensor proportional to P?.

\end{itemize}

\end{frame}